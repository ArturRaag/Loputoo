\begin{flushleft}

\section{Fehlbergi koefitsendid}
 
Selleks, et tabelis \ref{table_1} määratud koefitsentidele lihtsamini ligi pääseda, oleks mõistlik koefitsendid kirjutada andmemassiividena/maatriksitena. Nendele pääseb ligi indekseid kasutades.


Ehk:

\begin{equation}
A(k)=
\begin{pmatrix}
0\\
1/4\\
3/8\\
12/13\\
1\\
1/2
\end{pmatrix}
\end{equation}
 

\begin{equation}
B(k,l)=
\begin{pmatrix}
0&0&0&0&0\\

1/4&0&0&0&0\\

3/32 & 9/32 & 0 & 0 & 0 \\

1932/2197 & -7200/2197 & 7296/2197 & 0 & 0\\

439/216 & -8 & 3680/513 & -845/4104 & 0\\

-8/27 & 2 & -3544/2565 & 1859/4104 & -11/40 \\
\end{pmatrix}
\end{equation}

\begin{equation}
CH(k)=
\begin{pmatrix}
16/135\\
0\\
6656/12825\\
28561/56430\\
-9/50\\
2/55
\end{pmatrix}
\hspace{10mm}
CT(k)=
\begin{pmatrix}
1/360\\
0\\
-128/4275\\
-2197/75240\\
1/50\\
2/55
\end{pmatrix}
\end{equation}


Ka tõusudest $k_{i}$ loome massiivi:

\begin{equation}
k_{i}=
\begin{pmatrix}
k_{1}\\
k_{2}\\
k_{3}\\
k_{4}\\
k_{5}\\
k_{6}
\end{pmatrix}
\end{equation}  

Kuid meie mudelites esineb diferentsiaalvõrrandeid, mida lahendada soovime, rohkem kui üks. Seega, iga differentsiaalvõrrandi kohta, mida lahendame, tuleb lisada veel üks veerg $k$-sid. Ehk üldisemalt

\begin{equation}
k_{i,n}=
\begin{pmatrix}
k_{1,1} & k_{1,2} & ...& k_{1,n} \\
k_{2,1}& k_{2,2} & ...& k_{2,n}\\
k_{3,1}& k_{3,2} & ...& k_{3,n}\\
k_{4,1}& k_{4,2} & ...& k_{4,n}\\
k_{5,1}& k_{5,2} & ...& k_{5,n}\\
k_{6,1}& k_{6,2} & ...& k_{6,n}
\end{pmatrix}
\end{equation}  

kus $i$ on koefitsendi järjekorranumber(?) ja $n$ on võrrandi järjekorranumber.

Programmeerimiskeeles P5JS alustatakse iteratsioonide loendamist arvust 0, mitte 1, mida tasub ka programmi realiseerimisel meeles pidada.

\end{flushleft}
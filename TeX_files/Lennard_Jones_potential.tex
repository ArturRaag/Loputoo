\begin{flushleft}

\section{Lennard-Jones potentsiaal}
 
Lennard-Jonesi potentsiaal on defineeritud järgnevalt
\begin{equation} \label{eq_5.1}
V(r)=4 \epsilon \left[ \left( \dfrac{\sigma}{r} \right)^{12}- \left( \dfrac{\sigma}{r} \right)^{6} \right]
\end{equation}
kus $V$ on potentsiaal, $r$ on osakeste vaheline kaugus.

Osakeste vahelise kauguse $r$ moodul on defineeritud kui

\begin{equation} \label{eq_5.2}
%r=\overrightarrow{r_{i,j}}=\overrightarrow{r_{j}}-\overrightarrow{r_{i}}
r_{i,j}=\sqrt{(x_{j}-x_{i})^{2}+(y_{j}-y_{i})^{2}}
\end{equation}

kus $x_{j}$ ja $y_{j}$ on esimese ning $x_{i}$ ja $y_{i}$ on teise osakese asukoha vektori koordinaadid.

Kui rakendada $ \nabla $ operaatorit Lennard-Jonesi potentsiaalile \ref{eq_5.1}, siis saaksime jõu kahe osakese vahel, mis näeb välja nõnda:

\begin{equation} \label{eq_5.3}
\vec{F}=-\nabla V
\end{equation}

kus $V$ on Lennard-Jonesi potentsiaal \ref{eq_5.1}, ning

\begin{equation} \label{eq_5.4}
\nabla =\left[ \dfrac{ \delta }{\delta x} \hat{i}+\dfrac{\delta}{\delta y} \hat{j} \right]
\end{equation}



\begin{center}
\begin{tikzpicture}
\draw[help lines, color=gray!30, dashed] (-4.9,-4.9) grid (4.9,4.9);
\draw[-stealth,ultra thick] (-5,0)--(5,0) node[right]{$x$};
\draw[-stealth,ultra thick] (0,-5)--(0,5) node[above]{$y$};
\draw[-{Stealth[scale=2]}] (0,0)--(2,4) node[midway,left]{$r_{i}$};
\draw[-{Stealth[scale=2]}](0,0)--(4,4) node[midway,right]{$r_{j}$};
\filldraw[black] (2,4) circle (2pt);
\filldraw[black] (4,4) circle (2pt);
\draw[-{Stealth[scale=2]}](2,4)--(4,4) node [midway, above] {$r_{i,j}$};
\end{tikzpicture}
\end{center}

\textbf{Valemi tuletus KATSE 1:}



\[F=-4 \epsilon \left[ \sigma^{12} \dfrac{\delta}{\delta x} \left( r^{-12} \right) \hat{i}-\sigma^{6} \dfrac{\delta}{\delta x} (r^{-6}) \hat{i}+\sigma^{12} \dfrac{\delta}{\delta y} (r^{-12}) \hat{j} -\sigma^{6} \dfrac{\delta}{\delta y} (r^{-6}) \hat{j}\right]=... \]

Olgu $r$ hoopis $ r=\sqrt{x^{2}+y^{2}} $. Sellisel juhul tuleb

\[...= -4 \epsilon \left[ \sigma^{12} \dfrac{\delta}{\delta x} (x^{2}+y^{2})^{-6} \hat{i}-\sigma^{6} \dfrac{\delta}{\delta x} (x^{2}+y^{2})^{-3} \hat{i}+\sigma^{12} \dfrac{\delta}{\delta y} (x^{2}+y^{2})^{-6} \hat{j} -\sigma^{6} \dfrac{\delta}{\delta y} (x^{2}+y^{2})^{-3} \hat{j}\right]=...\]

Tuletiste arvutamisel võtame, et $(x^{2}+y^{2})=u$ (Chain Rule). Seega tuleb arvutada tuletised 
\[\dfrac{\delta}{\delta u} (u^{-6}) \cdot \dfrac{\delta}{\delta x} (x^{2}+y^{2}) \]

\[\dfrac{\delta}{\delta u} (u^{-6}) \cdot \dfrac{\delta}{\delta y} (x^{2}+y^{2}) \]

\[\dfrac{\delta}{\delta u} (u^{-3}) \cdot \dfrac{\delta}{\delta x} (x^{2}+y^{2}) \]

\[\dfrac{\delta}{\delta u} (u^{-3}) \cdot \dfrac{\delta}{\delta y} (x^{2}+y^{2}) \]

Need välja arvutades saame jätkata jõu valemi tuletamist

\[...=-4 \epsilon \left( \left[-12x \sigma^{12}(x^{2}+y^{2})^{-7}-6x \sigma^{6}(x^{2}+y^{2})^{-4} \right] \hat{i}+ \left[-12y \sigma^{12}(x^{2}+y^{2})^{-7}-6y \sigma^{6}(x^{2}+y^{2})^{-4} \right] \hat{j} \right) =...\]

\[...=-4 \epsilon \left(6x \sigma^{6} \left[-2 \sigma^{6}(x^{2}+y^{2})^{-7}- (x^{2}+y^{2})^{-4} \right] \hat{i}+6y \sigma^{6} \left[-2 \sigma^{6}(x^{2}+y^{2})^{-7}-(x^{2}+y^{2})^{-4} \right] \hat{j} \right) =...\]


\[...=-24  \epsilon \sigma^{6} \left(x  \left[-2 \sigma^{6}(x^{2}+y^{2})^{-7}- (x^{2}+y^{2})^{-4} \right] \hat{i}+y \left[-2 \sigma^{6}(x^{2}+y^{2})^{-7}-(x^{2}+y^{2})^{-4} \right] \hat{j} \right)  \]

Ehk:

\begin{equation} \label{eq_5.5}
F_{x}=-24 \epsilon \sigma^{6} x \left[ -2 \sigma^{6}(x^{2}+y^{2})^{-7}-(x^{2}+y^{2})^{-4} \right]
\end{equation}

\begin{equation} \label{eq_5.6}
F_{y}=-24 \epsilon \sigma^{6} y \left[ -2 \sigma^{6}(x^{2}+y^{2})^{-7}-(x^{2}+y^{2})^{-4} \right]
\end{equation}

\textbf{Valemi tuletus KATSE 2:}

\begin{equation} \label{eq_5.7}
\nabla V(r)=4 \epsilon \left[ \dfrac{\delta V}{\delta x_{i} }\hat{i}+\dfrac{\delta V}{\delta y_{i}}\hat{j} \right]
\end{equation}

\begin{equation}
V(r)=4 \epsilon \left[ \sigma^{12} r^{-12}-\sigma^{6} \cdot r^{-6} \right]
\end{equation}


\begin{equation} \label{eq_5.8}
\nabla V(r)=4 \epsilon \nabla \left[ \sigma^{12} \left[(x_{i}-x_{j})\hat{i} + (y_{i}-y_{j}) \hat{j} \right]^{-12} -\sigma^{6} \left[ (x_{i}-x_{j})\hat{i} +(y_{i}-y_{j})\hat{j} \right]^{-6} \right]
\end{equation}

Asendan vektoriaalse $r$ hoopis modulaarse $r$-iga, kuna jõud kahe osakese vahel suunast ei sõltu, sõltub vaid moodulist: $r=\sqrt{(x_{i}-x_{j})^{2}+(y_{i}-y_{j})^{2}}$.

\begin{equation} \label{eq_5.9}
\nabla V(r)=4 \epsilon \nabla \left[ \sigma^{12} \left[(x_{i}-x_{j})^{2}+(y_{i}-y_{j})^{2} \right]^{-6} -\sigma^{6} \left[ (x_{i}-x_{j})^{2}+(y_{i}-y_{j})^{2} \right]^{-3} \right]
\end{equation}


Tuleb arvutada järgmised osatuletised:

\begin{equation} \label{eq_5.11}
 \dfrac{\delta }{\delta x_{i}} \underbrace{((x_{i}-x_{j})^{2}+(y_{i}-y_{j})^{2})^{-6}}_{u} \cdot \hat{i}=\dfrac{\delta}{\delta u} (u)^{-6} \cdot  \dfrac{\delta }{\delta x_{i}} \left( (x_{i}-x_{j})^{2}+\cancel{(y_{i}-y_{j})^{2}} \right)\cdot \hat{i}
\end{equation}

Teised tuletised analoogselt...

\begin{equation} \label{eq_5.12}
\dfrac{\delta }{\delta y_{i}}((x_{i}-x_{j})^{2}+(y_{i}-y_{j})^{2})^{-6} \cdot \hat{j}
\end{equation}

\begin{equation} \label{eq_5.13}
 \dfrac{\delta }{\delta x_{i}}((x_{i}-x_{j})^{2}+(y_{i}-y_{j})^{2})^{-3} \cdot \hat{i}
\end{equation}

\begin{equation} \label{eq_5.14}
 \dfrac{\delta }{\delta y_{i}}((x_{i}-x_{j})^{2}+(y_{i}-y_{j})^{2})^{-3}\cdot \hat{j} 
\end{equation}

Alustame osatuletisega \ref{eq_5.11}

\[ \dfrac{\delta }{\delta x_{i}} \underbrace{((x_{i}-x_{j})^{2}+(y_{i}-y_{j})^{2})^{-6}}_{u}\cdot \hat{i} =\dfrac{\delta}{\delta u} (u)^{-6} \cdot  \dfrac{\delta }{\delta x_{i}} \left( (x_{i}-x_{j})^{2}+\cancel{(y_{i}-y_{j})^{2}} \right) \cdot \hat{i}=\]

\[= -6((x_{i}-x_{j})^{2}+(y_{i}-y_{j})^{2}) ^{-7} \cdot \dfrac{\delta}{\delta x_{i}} (\underbrace{x_{i}-x_{j}}_{v})^{2} \cdot \hat{i}=\]

\[ = -6 \cdot \left(  (x_{i} -x_{j})^{2}+(y_{i}-y_{j})^{2}\right)^{-7} \cdot \dfrac{\delta}{\delta v} (v)^{2} \cdot \dfrac{\delta}{\delta x_{i}}(x_{i}-x_{j}) \cdot \hat{i}= \]

\[=-6 \cdot \left(  (x_{i} -x_{j})^{2}+(y_{i}-y_{j})^{2}\right)^{-7} \cdot 2 (x_{i}-x_{j}) \cdot 1 \cdot \hat{i} =\]

\[=-12 \cdot \left(  (x_{i} -x_{j})^{2}+(y_{i}-y_{j})^{2}\right)^{-7} \cdot  (x_{i}-x_{j}) \cdot \hat{i} \]

Jätame \ref{eq_5.12} esialgu vahele, vaatleme hoopis \ref{eq_5.13}, siis saame jõu $F_{x}$.

\[ \dfrac{\delta }{\delta x_{i}}((x_{i}-x_{j})^{2}+(y_{i}-y_{j})^{2})^{-3} \cdot \hat{i}= -3((x_{i}-x_{j})^{2}+(y_{i}-y_{j})^{2}) ^{-4} \cdot \dfrac{\delta}{\delta x_{i}} (\underbrace{x_{i}-x_{j}}_{v})^{2} \cdot \hat{i}=\]

\[=-3((x_{i}-x_{j})^{2}+(y_{i}-y_{j})^{2}) ^{-4} \cdot 2 (x_{i}-x_{j}) \hat{i}= \]

\[=-6((x_{i}-x_{j})^{2}+(y_{i}-y_{j})^{2}) ^{-4} \cdot (x_{i}-x_{j}) \hat{i} \]

Kui panna viimased kaks tulemust kokku, siis saame, et $F_{x}$:

\[
F_{x}=4 \epsilon \sigma^{6} (12 \sigma^{6} \cdot \left(  (x_{i} -x_{j})^{2}+(y_{i}-y_{j})^{2}\right)^{-7} \cdot  (x_{i}+x_{j}) \cdot \hat{i}  -6((x_{i}-x_{j})^{2}+(y_{i}-y_{j})^{2}) ^{-4} \cdot (x_{i}-x_{j}) \hat{i})
\]

Kui veidi lihtsustada...

\begin{equation} \label{eq_5.15}
F_{x}=24 \epsilon \sigma^{6} [2\sigma^{6}((x_{i} -x_{j})^{2}+(y_{i}-y_{j})^{2})^{-7}-((x_{i}-x_{j})^{2}+(y_{i}-y_{j})^{2}) ^{-4}] \cdot (x_{i}-x_{j}) 
\end{equation}

Analoogselt $F_{y}$-i tuletisi \ref{eq_5.12} ja \ref{eq_5.14} arvutades saame:

\begin{equation} \label{eq_5.16}
F_{y}=24 \epsilon \sigma^{6} [2\sigma^{6}((x_{i} -x_{j})^{2}+(y_{i}-y_{j})^{2})^{-7}-((x_{i}-x_{j})^{2}+(y_{i}-y_{j})^{2}) ^{-4}] \cdot (y_{i}-y_{j})
\end{equation}

\section{Kiirendus}

Kui asendada valemid \ref{eq_5.15} ja \ref{eq_5.16} Newtoni II seadusesse, siis saaksime avaldada kiirendused.

\begin{equation} \label{eq_5.17}
F=ma
\end{equation}

\begin{equation} \label{eq_5.18}
24 \epsilon \sigma^{6} [2\sigma^{6}((x_{i} -x_{j})^{2}+(y_{i}-y_{j})^{2})^{-7}-((x_{i}-x_{j})^{2}+(y_{i}-y_{j})^{2}) ^{-4}] \cdot (x_{i}-x_{j})=ma
\end{equation}

Jagame mõlemad võrduse pooled miga

\begin{equation} \label{eq_5.19}
a_{x}=\dfrac{24 \epsilon \sigma^{6} [2\sigma^{6}((x_{i} -x_{j})^{2}+(y_{i}-y_{j})^{2})^{-7}-((x_{i}-x_{j})^{2}+(y_{i}-y_{j})^{2}) ^{-4}] \cdot (x_{i}-x_{j})}{m}

\end{equation}

\end{flushleft}